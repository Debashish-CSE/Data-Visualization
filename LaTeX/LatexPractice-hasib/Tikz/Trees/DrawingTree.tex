\documentclass{article}
\usepackage{forest}
\usepackage{tikz}
\title{Drawing Tree with Tikz}
\author{MD Hasib Mia}
\begin{document}
	\maketitle
	\tableofcontents
	\clearpage
	
	%---------------------------------------------------------------------------Create node------------------------------------------------
	
   \section{Create node and Color:}
   
   
	\subsection{use various shape}
	\subsubsection{Normal Node}
	\begin{forest}
		[A]
	\end{forest}
	
	

	\subsubsection{Circle shaped node:}
	\begin{forest}
		[A,circle,draw]
	\end{forest}
	
	
	\subsubsection{Rectangle shaped node:}
	\begin{forest}
		[A,rectangle,draw]
	\end{forest}
	
	
	
	\subsubsection{All shape in one tress:}
	\begin{forest}
		[A,circle,draw[B,rectangle,draw][C]]
	\end{forest}
	
	
	
	\subsection{Coloring of node:}
	\begin{forest}
		[A,circle,draw,color=black,fill=red!20]
	\end{forest}
	%---------------------------------------------------------------------------------------------------------------------------------------
	
	
	
	%---------------------------------------------------------Draw simple trees--------------------------------------------------------
	\section{Drawing Simple tree}
	
	
	\subsection{Parent Node: A and Child Node B,C:}
		% Here A----> is parent and B,C are child node.
		% Procedure: [p[c1][c2].......]
	\begin{forest}
		[A[B][C]]
	\end{forest}
	
	
	
	
	\subsection{More node with tree :}
     \begin{forest} 
     	[A[B[D[1][2]][E[3][4]]][C[F[5][6]][G[7][8]]]]
     \end{forest}
     
     
     
     \subsection{More node with tree with a single shape:}     
     \begin{forest} for tree={circle,draw}
     	[A[B[D[1][2]][E[3][4]]][C[F[5][6]][G[7][8]]]]
     \end{forest}
     
     
     
     
     \subsection{Draw Horizontal tree:}
     \begin{forest} for tree={grow'=0}
     	[A[B[D[1][2]][E[3][4]]][C[F[5][6]][G[7][8]]]]
     \end{forest}
     %-----------------------------------------------------------------------------------------------------------------------------
     
    
    %-----------------------------------------------------Shifting------------------------------------------------------------------- 
     \section{shifting of tree}
	\subsection{left side shift of Tree: (calign=first)}
	\begin{forest} 
		[A,calign=first[B[1][2]][C[3][4]]]
	\end{forest}
	
	
	
	
	\subsection{Right side shift of Tree: (calign=last)}
     \begin{forest} 
	[A,calign=last[B[1][2]][C[3][4]]]
    \end{forest}

%----------------------------------------------------------------------------------------------------------------------------------------

	\section{Align all first child node connected with immediate parent node by stright line:}
	% For this operation we have to use : for tree={calign=first}
	%For Set label of edge: edge label={node[midway, left] {Label 1}}
	 \begin{forest}
	 	for tree={calign=first}
		[A[B[D[1][2]][E[3][4]]][C[F[5][6]][G[7][8]]]]
	\end{forest}
	
	%-------------------------------------------------Practice--------------------------------------------------------------------
	
	
	\section{practice}
	\subsection{Practice-1:Vertical trees:}
	\begin{forest}
		[1[2[A][B[C][D]]][3[E][F[G][H]]]]
	\end{forest}
	
	
    \subsection{Practice-2:Horizontal trees:}
   \begin{forest} for tree={grow'=0}
	[1[2[A][B[C][D]]][3[E][F[G][H]]]]
    \end{forest}
    
    
    
    \subsection{practice-3: trees with rectangular node:}  
    \begin{forest}
    	for tree={draw}
    	[27[22[22[22][8]][17[14][17]]][27[9[3][9]][27[27][11]]]]
    \end{forest}

	
\end{document}
